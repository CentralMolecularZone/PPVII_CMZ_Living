\section{Summary and outlook}\label{sec:summary}

In the context of Galactic star formation, the CMZ is an environment like no other. It hosts the supermassive black hole, Sgr\,A$^{*}$, some of the closest and most massive young Galactic star clusters, the largest number of supernovae per unit volume, and the most concentrated reservoir of dense gas in the Milky Way. It is, furthermore, the only galactic nucleus in which it is currently possible to resolve the multi-scale physics of star formation down to the scales of protoplanetary discs. The recent discovery that the CMZ is underproducing stars relative to expectations based on its vast reservoir of dense gas has inspired a resurgence in observational and theoretical efforts to understand the star formation process in this complex environment, and has raised the important question of if (and how) the physics of star formation depends on local environmental conditions.

The factors responsible for the observed low present-day SFR of the CMZ are still the subject of intense scrutiny. Two forefront explanations have emerged, which may operate in unison. Either star formation is directly related to the macroscopic evolution of the system as a whole, possibly via a feedback-driven ``boom and bust'' duty cycle or via discrete large accretion events (\S\ref{sec:global}), and/or the extreme local ISM conditions (\S\ref{sec:cloudtodisc}) elevate the critical density for star formation (\S\ref{sec:environmentofstarformation}). Both explanations are supported by theoretical work and consistent with current observations, and both have implications that reach beyond the field of star and planet formation into that of galaxy formation and evolution. 
For the former of these scenarios, further theoretical work is needed to identify a specific variability mechanism and timescale. Higher-angular resolution observations of nearby galaxy centres will help to address whether the star formation in galactic nuclei occurs in discrete bursts or quasi-continuously. For the latter, tighter constraints on CMZ cloud properties, including the density structure, turbulence, and magnetic field strengths are needed to test predictions from star formation theory, in particular of a turbulence-regulated density threshold, and whether or not they hold under the extreme conditions in the CMZ.  

The observational future is bright. 
The observing capabilities offered by Large Programs with current (e.g. ALMA, VLA, SOFIA) and next-generation facilities such as the {\em James Webb Space Telescope} ({\em JWST}), Square Kilometre Array (SKA) and the next-generation VLA (ngVLA) will likely reshape our view of the CMZ. 
For example, the upcoming ALMA CMZ Exploration Survey, or  ``ACES'', will provide an unparalleled insight into the physical and kinematic structure of the CMZ, while polarisation measurements and deep, large spectral coverage surveys will allow us to further probe the magnetic field structure (\S\ref{sec:magneticfields}) and complex chemistry (\S\ref{sec:cloudchemistry}), respectively.
Combining with further observational and modelling work across the electromagnetic spectrum, from long-term VLBI maser monitoring to tracing X-ray flares, will add key constraints on the 3D geometry (\S\ref{sec:3d}).
Going beyond what was possible with {\it Spitzer} in the IR regime, the recently launched {\em JWST} will uncover hidden star formation in the CMZ, extending the YSO-counting methods used to determine the SFR in nearby clouds to the CMZ. 
This will directly address whether the low SFR in the CMZ is the result of a presently undetected low-mass YSO population, as well as whether the mass distribution of the YSOs is consistent with or different from a standard IMF (\S\ref{sec:starclusters}). 
High-angular-resolution observations of protostars will furthermore provide measurements of primordial binary statistics in the extreme environment of the CMZ (e.g., the Offner et al. chapter).
High-angular resolution observations (e.g.\ Fig~\ref{fig:high_res}) will also be critical in uncovering protoplanetary discs, with potentially important implications for our understanding of planet formation in extreme, cosmologically-representative environments (\S\ref{sec:ppds}).
A final key avenue for future observations is to go beyond the single sample size of our CMZ, and investigate the nuclear rings in other nearby galaxy centers with resolved observations from e.g. ALMA and {\em JWST}. 
Such observations will provide context and perspective for many of the open questions in our CMZ, including its geometry and the possibility of a ``boom and bust'' duty cycle for star formation in galaxy centers (\S\ref{subsec:SFR:starformationhistory}).

On the theoretical side, continued improvement to simulations of the inner regions of the Milky Way will be critical to understanding the global cycle of matter and energy in the CMZ. Global simulations  will help to understand the origin of turbulence in the CMZ gas (\S\ref{sec:turbulentdriving}), the transport of gas towards the central black hole (\S\ref{sec:cndinflow}), whether there are preferred locations for star formation in the CMZ (\S\ref{sec:sfhotspots}), and what drives its episodic nature (\S\ref{subsec:SFR:timeevolution}). Simulations that zoom-in at much higher resolution on individual molecular clouds will allow us to understand their formation self-consistently from the large-scale flow, and to follow the evolution of the clouds and of their embedded star formation as they orbit in the Galactic Centre. Zoom-in simulations will also probe the properties of dense gas, which is unresolved in current simulations, down to scales of $<0.1$ \pc, and post-processing of these simulations with radiative transfer tools will enable the creation of synthetic observations for direct comparisons with the plethora of observational data on the horizon. Finally, simulations of Milky Way-like galaxies in a cosmological context will provide further insight on how the evolution of galactic nuclei correlates with the evolution of their host galaxies across a wide range of galaxy properties.

Despite decades of observational and theoretical work, many foundational questions about the nature, context, and future of our CMZ remain unsettled. With decisive observational programs in the works and a resurgence of theoretical interest, we expect the next few years to be an exciting and productive time for CMZ research that will lead to major breakthroughs in our understanding of the star and planet formation process in extreme galactic environments and the role of galaxy centers in global evolution of galaxies.
\\

\noindent\textit{Acknowledgements: } The authors would like to thank Alyssa Goodman, and the other anonymous referee for their insightful comments that have helped to strengthen this review. We further would like to thank James Binney, Maïca Clavel, Filippo Fraternali, Dimitri Gadotti, Simon Glover, Jouni Kainulainen, Melanie Kaasinen, Allison Kirkpatrick, Diederik Kruijssen, Mark Krumholz, Adam Leroy, Steven Longmore, Xing (Walker) Lu, Mattis Magg, Betsy Mills, Desika Narayanan, Tomoharu Oka, Gabriele Ponti, Miguel Querejeta, Rainer Schödel, Yoshiaki Sofue, and Jiayi Sun for helpful discussions and comments. 
We thank the speakers of the CMZoom talk series (\url{https://sites.google.com/view/cmzsftalkseries/home}), Thushara Pillai, Kunihiko Tanaka, Pei-Ying Hsieh, Farhad Yusef-Zadeh, Betsy Mills, Jesus Salas, Mark Krumholz, Alvaro Sanchez-Monge, Francisco Nogueras Lara, Maïca Clavel, Mark Morris, Matt Hosek, and Melisse Bonfand-Caldeira.

